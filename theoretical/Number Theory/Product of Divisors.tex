\subsection{Product of Divisors}

Let the product and number of divisors when only considering the first $i$ prime factors be $P_i$ and $C_i$ respectively. The answer will be $P_N$.

$$P_i = P_{i - 1}^{k_{i} + 1} \left( x_{i}^{k_{i}(k_i + 1)/2} \right)^{C_{i - 1}}$$

Again, we can calculate each $P_i$ using fast exponentiation in $\mathcal O(N \log(\max(k_i)))$ time, but there's a catch! It might be tempting to use $C_{i - 1}$ from your previously-calculated values in part 1 of this problem, but those values will yield wrong answers.\\
\hfill \break
This is because $a^b \not \equiv a^{b \bmod p} \pmod{p}$ in general. However, by Fermat's little theorem, $a^b \equiv a^{b \bmod (p - 1)} \pmod{p}$ for prime $p$, so we can just store $C_i$ modulo $10^9 + 6$ to calculate $P_i$.